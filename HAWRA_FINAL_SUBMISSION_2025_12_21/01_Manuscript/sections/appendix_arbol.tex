\section{Formal EBNF Grammar}

The \texttt{ARBOL} language is defined by the following Extended Backus-Naur Form (EBNF) grammar, which supports both high-level quantum abstractions and direct biological stimulus control:

\begin{verbatim}
program          = { statement } ;
statement        = declaration | instruction | circuit_def | run_stmt ;
declaration      = "qubit" identifier [ "[" integer "]" ] ;
instruction      = gate_op | stimulus_op | measurement | control_flow ;
gate_op          = ( "h" | "x" | "z" | "cnot" | "ry" | "rx" ) identifier_list ;
stimulus_op      = "apply" stimulus_type "(" param_list ")" "to" identifier ;
measurement      = [ identifier "=" ] "measure" identifier ;
control_flow     = "wait_metabolic" "(" duration ")" | "for" identifier "in" range ;
circuit_def      = "circuit" identifier "(" [ param_list ] ")" "{" { statement } "}" ;
run_stmt         = "run" identifier "(" [ arg_list ] ")" ";" ;
\end{verbatim}

\section{BSIM v0.4 Opcode Table}

The following table lists the low-level instructions generated by the ARBOL compiler for the Bio-Simulator and the BioOS kernel.

\begin{table}[ht]
\centering
\begin{tabular}{|l|l|l|}
\hline
Opcode & Description & Parameters \\
\hline
\texttt{GATE\_APPLY} & Applies a logical quantum gate & \texttt{gate}, \texttt{targets} \\
\texttt{STIMULUS\_APPLY} & Triggers a physical biological stimulus & \texttt{type}, \texttt{intensity}, \texttt{duration} \\
\texttt{MEASURE} & Reads qubit state (via fluorescence/bioluminescence) & \texttt{target}, \texttt{register} \\
\texttt{RUN\_START} & Telemetry marker for circuit execution start & \texttt{circuit\_name} \\
\texttt{RUN\_END} & Telemetry marker for circuit execution end & \texttt{circuit\_name} \\
\texttt{METABOLIC\_YIELD} & Idle cycle for ATP pool recovery & \texttt{duration\_ms} \\
\texttt{SYNC\_PTP} & Synchronizes multi-leaf clock via PTP & \texttt{node\_id} \\
\texttt{ERR\_METABOLIC} & Error code for physiological distress & \texttt{severity}, \texttt{code} \\
\hline
\end{tabular}
\caption{Exhaustive list of low-level instructions for the HAWRA Bio-Simulator.}
\end{table}

\section{Case Study: Biological Quantum Walk (BQW)}

To demonstrate the expressive power of ARBOL, we present the implementation of a Discrete-Time Quantum Walk (DTQW) on a 1D biological lattice. This algorithm is used to model the efficient transport of excitons within the Light-Harvesting Complexes.

\begin{lstlisting}[language=Python, caption=Implementation of a Biological Quantum Walk in ARBOL]
# Lattice initialization: 8 biological qubits (P700)
qubit lattice[0..7]

# Coin operation: Hadamard gate applied to the coin qubit
circuit coin_flip(q):
    h q
    apply recovery(mode=far_red) to q

# Conditional shift: Excitonic transport between reaction centers
circuit shift(l):
    for i in 0..6:
        cnot l[i], l[i+1]
        wait_metabolic(10ps)

# Main BQW loop: 10 steps
run initialization(lattice)
for step in 1..10:
    run coin_flip(lattice[0])
    run shift(lattice)
    measure lattice -> results
\end{lstlisting}

\section{Coherence Constraints and Timing Specifications}

To ensure computational validity, the ARBOL compiler enforces strict timing constraints on the generated instruction stream. Specifically, the interval $\Delta t$ between consecutive \texttt{GATE\_APPLY} operations is constrained to:

\begin{equation}
\Delta t < \frac{T_2}{10} \cdot \eta_{shield}
\end{equation}

where $T_2$ is the estimated transverse relaxation time and $\eta_{shield}$ is the efficiency of the silica-biomineralization module. If the logical flow necessitates a delay exceeding this threshold, the compiler automatically injects a \textit{Phase Recalibration} sequence to compensate for the anticipated dephasing. Furthermore, every 100 logical operations, a \texttt{METABOLIC\_YIELD} is mandatory to prevent thermal runaway and local nutrient depletion in the thylakoid membrane.
