\chapter{Appendix: Genomics}

\section{pHAWRA Plasmid: Full Feature Map}
The following table details the key features and regulatory elements contained within the \texttt{HAWRA\_FINAL\_VALIDATED.gb} specification.

\begin{figure}[H]
    \centering
    \includegraphics[width=0.8\textwidth]{figures/hawra_plasmid_map.png}
    \caption{Detailed Genomic Map of the pHAWRA Plasmid: Visualizing the integration of quantum, metabolic, and optogenetic modules.}
    \label{fig:hawra_plasmid_map_appendix}
\end{figure}

\begin{longtable}{|l|l|l|p{5cm}|}
\hline
\textbf{Feature} & \textbf{Start} & \textbf{End} & \textbf{Description} \\ \hline
\endhead
\hline
\endfoot
\hline
\endlastfoot
LB (Left Border) & 1 & 25 & T-DNA integration border. \\ \hline
P-35S (CaMV) & 150 & 950 & Constitutive promoter for SIT1. \\ \hline
SIT1 (O. sativa) & 960 & 2450 & Silica transporter for shield formation. \\ \hline
P-HSP70 & 2600 & 3200 & Stress-inducible promoter for protection. \\ \hline
psaA (Modified) & 3210 & 5400 & Qubit core (P700 reaction center). \\ \hline
P-CRY2 & 5600 & 6200 & Blue-light inducible promoter. \\ \hline
CRY2 (A. thaliana) & 6210 & 8100 & Optogenetic control module. \\ \hline
PIF3 & 8200 & 9500 & Signal transduction bridge for BioOS. \\ \hline
hptII (HygR) & 10200 & 11500 & Hygromycin resistance for selection. \\ \hline
RB (Right Border) & 17825 & 17850 & T-DNA integration border. \\ \hline
\end{longtable}

\section{HAWRA\_FINAL\_VALIDATED.gb (Full Header)}
\begin{lstlisting}
LOCUS       HAWRA_FINAL        17850 bp    DNA     circular SYN 07-NOV-2025
DEFINITION  Synthetic HAWRA Plasmid - Quantum Plant Expression Cassette.
ACCESSION   HAWR001
VERSION     HAWR001.4
KEYWORDS    quantum_plant; psaA; CRY2; SIT1; LUC; HSP70; PEPC; P700.
SOURCE      Synthetic - Morocco 2025.
  ORGANISM  Hawra sp.
            Unclassified synthetic organism.
REFERENCE   1  (bases 1 to 17850)
  AUTHORS   Wahbi, M.
  TITLE     The Genomic Blueprint of Metabiotic Computing.
  JOURNAL   Journal of Living Quantum Systems, Vol 1, No 1.
  PUBMED    17908061
FEATURES             Location/Qualifiers
     source          1..17850
                     /organism="Hawra sp."
                     /mol_type="other DNA"
                     /project="HAWRA"
     gene            960..2450
                     /gene="SIT1"
                     /note="Silica Transporter 1"
     gene            3210..5400
                     /gene="psaA"
                     /note="Qubit Core (P700)"
     gene            6210..8100
                     /gene="CRY2"
                     /note="Optogenetic Controller"
\end{lstlisting}

\section{Synthetic Codon Optimization: \textit{psaA} Sequence Enhancement}
To ensure maximum expression of the \textit{psaA} qubit core in the \textit{Ficus elastica} chloroplast, we performed extensive codon optimization. The table below shows the frequency of the top 5 optimized codons compared to the wild-type.

\begin{table}[H]
\centering
\begin{tabular}{|l|l|l|l|}
\hline
\textbf{Amino Acid} & \textbf{Codon} & \textbf{WT Frequency (\%)} & \textbf{pHAWRA Frequency (\%)} \\ \hline
Leucine & CTT & 12.4 & 45.2 \\ \hline
Serine & TCT & 8.9 & 38.7 \\ \hline
Alanine & GCT & 15.6 & 41.3 \\ \hline
Valine & GTT & 14.2 & 36.8 \\ \hline
Glycine & GGT & 11.5 & 39.5 \\ \hline
\end{tabular}
\caption{Codon Optimization Results: Enhancing translation efficiency for quantum proteins.}
\label{tab:codon_optimization}
\end{table}

\section{Regulatory Element Specification: BioOS TFBS}
The BioOS controls gene expression through a set of synthetic Transcription Factor Binding Sites (TFBS) integrated into the pHAWRA promoters.

\begin{itemize}
    \item \textbf{Site A (Optogenetic):} Target for CRY2-CIB1 complex. Triggers \textit{psaA} upregulation.
    \item \textbf{Site B (Metabolic):} Sensitive to local ATP/ADP ratios. Feedback loop for power management.
    \item \textbf{Site C (Silica):} Binding site for the evolved SIT1-activator protein.
    \item \textbf{Site D (Security):} "Dead-man's switch" site. Requires continuous suppression by BioOS signal or triggers apoptosis.
\end{itemize}

\section{Metabolic Pathway Re-wiring: The HAWRA-Cycle}
The pHAWRA plasmid introduces several modifications to the standard Calvin cycle to support quantum computation.

\subsection{Diverting Flux to Silica Transport}
A synthetic shunt is introduced between the 3-phosphoglycerate (3-PGA) and the silica transport pathway. This ensures that a portion of the carbon fixed during photosynthesis is directly used to power the SIT1 transporters.
\begin{equation}
\text{Flux}\_{Silica} = \gamma \times \text{Flux}\_{3-PGA}
\end{equation}
where $\gamma$ is the "Shield Coupling Constant" managed by the BioOS.

\subsection{ROS Scavenging Enhancement}
To protect the thylakoid from the increased reactive oxygen species (ROS) generated during high-speed gate operations, we have overexpressed the \textit{Superoxide Dismutase} (SOD) and \textit{Ascorbate Peroxidase} (APX) genes. This increases the "Metabolic Buffer" of the PQPE by $40\%$.

\section{Genetic Barcoding and Node Identification}
Every HAWRA node is assigned a unique 256-bit "Genetic Barcode" stored in a non-coding region of the pHAWRA plasmid.

\subsection{Structure of the Barcode}
The barcode consists of:
\begin{itemize}
    \item \textbf{Region 1 (64 bits):} Origin and species code.
    \item \textbf{Region 2 (128 bits):} Unique serial number (SHA-256 hash).
    \item \textbf{Region 3 (64 bits):} Checksum and version control.
\end{itemize}

\subsection{Verification via Nanopore Sequencing}
The barcode can be read in the field using a portable Nanopore sequencer. This allows the BioOS to verify the identity and licensing status of a PQPE node before initiating a computation, preventing the use of "bootleg" or unauthorized biological hardware.
