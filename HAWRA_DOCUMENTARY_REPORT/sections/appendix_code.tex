\chapter{Appendix B: Code and Driver Specifications}

\section{BioOS Hardware Abstraction Layer (HAL) - C++ Snippet}
The following C++ code represents the core of the BioOS HAL, responsible for translating logical gate operations into optical stimuli.

\begin{lstlisting}[language=C++]
// BioOS_HAL_P700.cpp
#include <bioos/hal.h>
#include <bioos/optics.h>

class P700_Driver {
public:
    void apply_hadamard(uint32_t qubit_id) {
        // Retrieve physical coordinates from the Metabiotic Registry
        auto coord = registry.get_qubit_coords(qubit_id);
        
        // Shape the pulse: 450nm, 10fs envelope
        Pulse p = pulse_shaper.create_pulse(450.0, 10.0e-15);
        p.set_intensity(1.2e3); // kW/cm^2
        
        // Execute stimulus via the Laser Control Unit (LCU)
        lcu.fire_at(coord, p);
        
        // Update metabolic telemetry
        telemetry.log_atp_draw(0.5e-15); // Joules
    }
};
\end{lstlisting}
\section{BSIM Assembler - Python Implementation}
The BSIM assembler translates ARBOL assembly into the binary format used by the BioOS kernel.

\begin{lstlisting}[language=Python]
# bsim_assembler.py
import struct

class BSIMAssembler:
    def __init__(self):
        self.opcode_map = {
            "GATE_H": 0x01,
            "GATE_X": 0x02,
            "GATE_Z": 0x03,
            "MEASURE": 0x04,
            "WAIT_MET": 0x05
        }

    def assemble_instruction(self, instr, params):
        opcode = self.opcode_map.get(instr, 0x00)
        # Format: [Opcode (1b), Qubit (1b), Intensity (4b), Duration (4b), Padding (6b)]
        return struct.pack("<BBff6x", opcode, params['qubit'], params['intensity'], params['duration'])

    def build_file(self, instructions):
        header = b"BSIM\x01\x00\x00\x00" # Magic bytes + version
        body = b"".join([self.assemble_instruction(i, p) for i, p in instructions])
        return header + body
\end{lstlisting}

\section{Bio-Stochastic Gradient Descent (Bio-SGD) - PyTorch implementation}
The Bio-SGD algorithm incorporates metabolic stress into the loss function, ensuring that the model weights are optimized for both accuracy and biological sustainability.

\begin{lstlisting}[language=Python]
# bio_sgd.py
import torch
from torch.optim import Optimizer

class BioSGD(Optimizer):
    def __init__(self, params, lr=1e-3, alpha=0.1, beta=0.05):
        defaults = dict(lr=lr, alpha=alpha, beta=beta)
        super(BioSGD, self).__init__(params, defaults)

    @torch.no_grad()
    def step(self, closure=None, metabolic_stress=0.0):
        loss = None
        if closure is not None:
            with torch.enable_grad():
                loss = closure()

        for group in self.param_groups:
            alpha = group['alpha']
            beta = group['beta']
            
            for p in group['params']:
                if p.grad is None:
                    continue
                
                d_p = p.grad
                # Add metabolic regularization term
                # Penalty increases as metabolic_stress (e.g., ROS levels) rises
                metabolic_penalty = beta * metabolic_stress * p.data
                
                p.add_(d_p + metabolic_penalty, alpha=-group['lr'])

        return loss
\end{lstlisting}

\section{Conclusion}
The code provided in this appendix is intended for research and educational purposes. The Move37 Initiative maintains a private, production-grade repository with full testing and validation suites.

\section{ARBOL Standard Library: \texttt{bio\_math.arb}}
The \texttt{bio\_math} library provides high-level mathematical functions optimized for the excitonic substrate.

\begin{lstlisting}
// bio_math.arb
module bio_math;

export function quantum_fourier_transform(register reg) {
    for (i = 0; i < reg.length; i++) {
        gate H(reg[i]);
        for (j = i + 1; j < reg.length; j++) {
            gate CP(reg[j], reg[i], PI / (2^(j-i)));
        }
    }
}

export function estimate_metabolic_cost(circuit c) {
    return c.total_gates * 1.2e-15; // Mean cost per gate
}
\end{lstlisting}

\section{BioOS Metabiotic Scheduler - Python Controller}
The high-level scheduler resides on the external Jetson device and manages the long-term health of the PQPE node.

\begin{lstlisting}[language=Python]
# scheduler.py
import bioos_api as bio

def run_scheduler_loop():
    plant = bio.connect_node("PQPE_01")
    
    while True:
        status = plant.get_health_status()
        
        if status.starch_level < 0.2:
            print("CRITICAL: Starch low. Entering Recharge Mode.")
            plant.set_mode(bio.MODE_RECHARGE)
            continue
            
        job_queue = plant.get_pending_jobs()
        for job in job_queue:
            if job.priority == bio.PRIO_HIGH or status.is_diurnal:
                # Calculate Metabolic Feasibility Score (Sm)
                sm = status.atp_level / (job.estimated_cost * (1 + status.ros_stress))
                
                if sm > 1.0:
                    plant.execute(job)
                else:
                    plant.delay(job, reason="Metabolic Constraint")
\end{lstlisting}

\section{Stochastic Lindblad Solver - Python/JAX Implementation}
To achieve high-performance simulations of the excitonic dynamics, we use a custom Lindblad solver implemented in JAX for GPU acceleration.

\begin{lstlisting}[language=Python]
import jax.numpy as jnp
from jax import jit, vmap

@jit
def lindblad_evolution(rho, H, L_ops, dt):
    """Calculates one step of Lindblad evolution."""
    # Unitary part
    drho = -1j * (H @ rho - rho @ H)
    
    # Dissipative part
    for L in L_ops:
        drho += L @ rho @ L.conj().T - 0.5 * (L.conj().T @ L @ rho + rho @ L.conj().T @ L)
    
    return rho + drho * dt

def simulate_pqpe(params, t_max, dt):
    """Simulates the P700 exciton dynamics."""
    # Initialize state and Hamiltonian
    rho = jnp.zeros((7, 7), dtype=jnp.complex64)
    rho = rho.at[0, 0].set(1.0) # Initial excitation at site 0
    H = construct_hamiltonian(params)
    L_ops = construct_jump_operators(params)
    
    # Time loop
    for t in jnp.arange(0, t_max, dt):
        rho = lindblad_evolution(rho, H, L_ops, dt)
    return rho
\end{lstlisting}

\section{Living Kernel Driver: Root-Fungal Interface (Rust)}
The low-level driver for the Mycorrhizal Interface Module (MIM) is written in Rust for safety and performance.

\begin{lstlisting}[language=Rust]
// mim_driver.rs
use bioos_hal::{I2cBus, Pin};

pub struct MimDriver {
    bus: I2cBus,
    reset_pin: Pin,
}

impl MimDriver {
    pub fn init(&mut self) -> Result<(), Error> {
        self.reset_pin.set_high();
        self.bus.write(MIM_ADDR, &[REG_INIT, 0x01])?;
        Ok(())
    }

    pub fn send_packet(&mut self, packet: &MtpPacket) -> Result<(), Error> {
        // Translate MTP packet to calcium wave pulses
        let pulses = packet.to_pulses();
        for pulse in pulses {
            self.bus.write(MIM_ADDR, &[REG_PULSE, pulse.intensity])?;
            delay_ms(pulse.duration);
        }
        Ok(())
    }

    pub fn read_telemetry(&self) -> MimTelemetry {
        let mut buf = [0u8; 4];
        self.bus.read(MIM_ADDR, REG_TELEMETRY, &mut buf).unwrap();
        MimTelemetry::from_bytes(buf)
    }
}
\end{lstlisting}

\section{pHAWRA Plasmid: Genetic Firewall Logic}
The Genetic Firewall is implemented as a set of synthetic transcription factor decoys that compete for binding sites with endogenous plant signals.

\begin{itemize}
    \item \textbf{Decoy A:} Competes with ABA-induced stress signals to prevent accidental shutdown of the silica transporters.
    \item \textbf{Decoy B:} Buffers the $CRY2$ expression to prevent over-sensitivity to ambient blue light.
\end{itemize}

\section{BSIM Bytecode Format v1.0}
The BSIM file is a binary format composed of fixed-size frames (64 bytes).

\begin{itemize}
    \item \textbf{Bytes 0-3:} Timestamp (microseconds from start).
    \item \textbf{Bytes 4-7:} Stimulus ID (0=Photon, 1=Thermal, 2=Chemical).
    \item \textbf{Bytes 8-15:} Intensity (64-bit float).
    \item \textbf{Bytes 16-23:} Duration (64-bit float).
    \item \textbf{Bytes 24-31:} Target X Coordinate.
    \item \textbf{Bytes 32-39:} Target Y Coordinate.
    \item \textbf{Bytes 40-63:} Reserved for Error Correction Codes (ECC).
\end{itemize}

\section{ARBOL Example: Quantum Entanglement and Metabolic Awareness}
The following ARBOL script demonstrates the creation of a Bell state (entanglement) between two P700 qubits in a living leaf, incorporating metabolic checks and biological error handling.

\begin{lstlisting}
// HAWRA ARBOL v1.0
// Project: HAWRA_POC_VALIDATION
// Author: ARBOL-Compiler-v1.0.2

import "bio_std"

qubit q0 @ leaf[0].thylakoid[142];
qubit q1 @ leaf[0].thylakoid[143];

circuit BellState() {
    // 1. Pre-computation Metabolic Check
    // Ensures CO2 flux is sufficient for exciton stability
    wait_metabolic(flux_co2 > 12.0);
    
    // 2. Silica Shield Stabilization
    // Pulse-triggering SIT1 for peak dielectric efficiency
    apply photon(450nm, 150ms) to leaf[0];
    
    // 3. Quantum Logic Operations
    try {
        if_stable {
            H(q0);            // Create superposition
            CNOT(q0, q1);     // Entangle with neighbor
        }
    } catch (DecoherenceStorm e) {
        // Fallback: Flush state to classical anthocyanin buffer
        apply thermal(28C, 10s) to leaf[0];
        log("Bio-Error: Decoherence detected. State flushed.");
    }
    
    // 4. Readout Phase
    // Measurement via P700 absorption spectroscopy
    measure q0 -> register[0];
    measure q1 -> register[1];
}
\end{lstlisting}

\section{HAWRA-Sim: Lindblad Solver Core (Python)}
The following Python snippet shows the core of the quantum solver, implementing the Lindblad evolution of the density matrix.

\begin{lstlisting}[language=Python]
import numpy as np
from scipy.linalg import expm

def evolve_lindblad(rho_0, H, L_ops, gamma, dt, steps):
    """
    Evolves the density matrix rho according to the Lindblad equation.
    """
    rho = rho_0
    for _ in range(steps):
        # Hamiltonian part
        d_rho = -1j * (H @ rho - rho @ H)
        
        # Dissipative part
        for L, g in zip(L_ops, gamma):
            d_rho += g * (L @ rho @ L.conj().T - 0.5 * (L.conj().T @ L @ rho + rho @ L.conj().T @ L))
            
        rho += d_rho * dt
        # Normalize to maintain Tr(rho) = 1
        rho /= np.trace(rho)
    return rho

# Example: H-Gate pulse on P700
H_hadamard = np.array([[0, 1], [1, 0]]) * np.pi / (2 * 50e-12) # 50ps pulse
\end{lstlisting}

\section{Metabolic Engine: ODE System (Python)}
The metabolic engine uses a system of ODEs to model the plant's physiological state.

\begin{lstlisting}[language=Python]
def metabolic_system(y, t, light_intensity, T_ambient):
    """
    ODE system for Ficus elastica metabolism.
    y = [P700_conc, starch_conc, silica_thickness]
    """
    p700, starch, silica = y
    
    # 1. P700 Production (Light-dependent)
    k_prod = 0.1 * (light_intensity / (0.5 + light_intensity))
    k_deg = 0.02
    dp700 = k_prod - k_deg * p700

    # 2. Starch Accumulation
    k_photo = 0.05 * light_intensity * p700
    k_resp = 0.01 * np.exp(0.1 * T_ambient) # Arrhenius-like respiration
    dstarch = k_photo - k_resp

    # 3. Silica Deposition (Active Transport)
    k_si = 0.005 * starch / (1.0 + starch)
    dsilica = k_si

    return [dp700, dstarch, dsilica]
\end{lstlisting}

\section{BioOS Scheduler Core (Rust)}
The performance-critical parts of the BioOS scheduler are implemented in Rust to ensure memory safety and zero-cost abstractions during high-frequency gate orchestration.

\begin{lstlisting}[language=Rust]
// bioos-scheduler/src/core.rs
use bioos_hal::{Gate, Qubit, MetabolicState};

pub struct Scheduler {
    job_queue: Vec<Job>,
    metabolic_limit: f64,
}

impl Scheduler {
    pub fn schedule_next(&mut self, state: &MetabolicState) -> Option<Job> {
        // Sort jobs by Metabiotic Priority Score (MPS)
        self.job_queue.sort_by(|a, b| {
            let mps_a = a.priority as f64 * state.atp_availability;
            let mps_b = b.priority as f64 * state.atp_availability;
            mps_b.partial_cmp(&mps_a).unwrap()
        });

        if let Some(job) = self.job_queue.pop() {
            if self.is_safe(&job, state) {
                return Some(job);
            }
        }
        None
    }

    fn is_safe(&self, job: &Job, state: &MetabolicState) -> bool {
        // Prevent metabolic exhaustion
        let predicted_cost = job.gates.len() as f64 * 0.05;
        state.starch_reserve > predicted_cost && state.leaf_temp < 35.0
    }
}
\end{lstlisting}

\section{Living Kernel Driver (Rust)}
The Living Kernel Driver is the lowest level of the BioOS, interacting directly with the optogenetic stimulators and the physiological sensors.

\begin{lstlisting}[language=Rust]
// bioos-kernel/src/driver.rs
use bioos_hal::{Stimulus, Sensor, Frequency};

pub struct LivingDriver {
    stimulator: Box<dyn Stimulus>,
    sensor: Box<dyn Sensor>,
}

impl LivingDriver {
    pub fn new() -> Self {
        LivingDriver {
            stimulator: Box::new(BlueLaser::new(Frequency::from_nm(450.0))),
            sensor: Box::new(FluorescenceSensor::new()),
        }
    }

    pub fn execute_gate(&self, pulse_width: f64, intensity: f64) -> Result<f64, DriverError> {
        // 1. Check Pre-pulse Metabolic State
        let baseline = self.sensor.read_fluorescence()?;
        
        // 2. Fire the Optogenetic Stimulus
        self.stimulator.pulse(pulse_width, intensity)?;
        
        // 3. Measure Post-pulse Response
        let response = self.sensor.read_fluorescence()?;
        
        // 4. Calculate Delta (Quantum Feedback)
        Ok(response - baseline)
    }
}
\end{lstlisting}

\section{Stochastic Lindblad Solver (Python/JAX)}
This solver uses JAX for high-performance, differentiable simulation of the quantum state under biological noise.

\begin{lstlisting}[language=Python]
# jax_lindblad.py
import jax.numpy as jnp
from jax import grad, jit

@jit
def lindblad_step(rho, H, L_ops, gamma, dt):
    # Hamiltonian evolution
    d_rho = -1j * (H @ rho - rho @ H)
    
    # Stochastic Lindblad dissipation
    for L, g in zip(L_ops, gamma):
        # Adding a noise term to the dissipation rate
        g_noisy = g + 0.1 * jnp.random.normal() 
        d_rho += g_noisy * (L @ rho @ L.conj().T - 0.5 * (L.conj().T @ L @ rho + rho @ L.conj().T @ L))
    
    return rho + d_rho * dt

# Compute gradient of fidelity w.r.t pulse intensity
def fidelity_loss(intensity, target_rho):
    H = intensity * H_base
    rho_final = evolve_full(rho_0, H, L_ops, gamma, T)
    return 1.0 - jnp.trace(rho_final @ target_rho).real

grad_fidelity = grad(fidelity_loss)
\end{lstlisting}

\section{Mycorrhizal Transmission Protocol (MTP v1.1)}
The MTP protocol defines the packet structure for weight updates sent through the fungal network.

\begin{lstlisting}
Packet MTP_v1 {
    Header: [0xHA, 0x52, 0x41], // H-A-R magic bytes
    NodeID: uint32,
    Sequence: uint16,
    PayloadType: 0x01 (WeightUpdate),
    Payload: [float32; 1024],    // Compressed weights
    Checksum: uint32
}
\end{lstlisting}
The BioOS scheduler is implemented in Rust to ensure memory safety and low-latency response to metabolic interrupts.

\begin{lstlisting}
// BioOS Kernel: Metabolic-Aware Scheduler
// Implements the EDF-MA (Earliest Deadline First with Metabolic Awareness) algorithm.

struct Task {
    id: u32,
    priority: u8,
    deadline: u64,
    atp_cost: f32,
    co2_requirement: f32,
}

struct Scheduler {
    ready_queue: Vec<Task>,
    metabolic_state: PlantState,
}

impl Scheduler {
    fn schedule(&mut self) -> Option<Task> {
        // 1. Check current ATP levels
        if self.metabolic_state.atp_level < 0.2 {
            println!("NMI: ATP Critical. Suspending all tasks.");
            return None;
        }

        // 2. Filter tasks based on available CO2
        let available_tasks: Vec<Task> = self.ready_queue
            .iter()
            .filter(|t| t.co2_requirement < self.metabolic_state.current_co2_flux)
            .cloned()
            .collect();

        // 3. Select task with earliest deadline
        available_tasks.into_iter().min_by_key(|t| t.deadline)
    }
}
\end{lstlisting}
