\chapter{Standard Operating Procedures (SOP)}

\section{SOP-001: Transformation and Regeneration of the PQPE}
The creation of a PQPE (Phyto-synthetic Quantum Processing Entity) requires precise laboratory protocols to ensure high transformation efficiency and robust regeneration of the living substrate.

\subsection{Protoplast Isolation and Preparation}
\begin{enumerate}
    \item \textbf{Leaf Selection:} Select young, fully expanded leaves from a healthy, 2-year-old \textit{Ficus elastica} donor plant.
    \item \textbf{Surface Sterilization:} Wash leaves in 70\% ethanol for 1 minute, followed by 2\% sodium hypochlorite for 15 minutes. Rinse five times in sterile distilled water.
    \item \textbf{Enzymatic Digestion:} Slice leaves into 1mm strips and incubate in a solution of 1.5\% Cellulase R-10 and 0.4\% Macerozyme R-10 for 12 hours at $25^\circ$C in the dark.
    \item \textbf{Purification:} Filter the suspension through a $40 \mu m$ mesh and centrifuge at 100g for 5 minutes. Resuspend the protoplast pellet in W5 solution.
\end{enumerate}

\subsection{PEG-Mediated Transformation}
\begin{enumerate}
    \item \textbf{DNA Addition:} Add $20 \mu g$ of purified pHAWRA plasmid (\texttt{HAWRA\_FINAL\_VALIDATED.gb}) to $200 \mu L$ of protoplast suspension ($10^5$ cells/mL).
    \item \textbf{PEG Treatment:} Add an equal volume of 40\% PEG 4000 and incubate for 20 minutes at room temperature.
    \item \textbf{Wash and Recovery:} Slowly dilute the PEG with W5 solution, centrifuge, and resuspend the protoplasts in liquid WI medium. Incubate for 24 hours in the dark for initial gene expression.
\end{enumerate}

\subsection{Callus Induction and Shoot Regeneration}
\begin{enumerate}
    \item \textbf{Plating:} Plate the transformed protoplasts on MS medium supplemented with BAP (2.0 mg/L) and Kinetin (0.5 mg/L).
    \item \textbf{Selection:} After 1 week, transfer to medium containing 50 mg/L Hygromycin B to select for transgenic cells.
    \item \textbf{Subculturing:} Subculture developing calli every 3 weeks until green shoot primordia appear.
    \item \textbf{Rooting:} Transfer shoots to half-strength MS medium with 0.1 mg/L IBA to induce rooting.
\end{enumerate}

\section{SOP-006: Algal Bio-Reactor Setup and PQPE Cultivation}
While HAWRA-Ficus is used for large-scale forest deployment, HAWRA-Algae is optimized for high-throughput laboratory computing in photobioreactors.

\subsection{Media Preparation and Inoculation}
\begin{enumerate}
    \item \textbf{Medium Formulation:} Prepare BG-11 medium supplemented with 100 $\mu$M sodium metasilicate ($Na\_2SiO\_3 \cdot 9H\_2O$) for silica shield support.
    \item \textbf{Sterilization:} Autoclave at 121$^\circ$C for 20 minutes.
    \item \textbf{Inoculation:} Inoculate with transformed \textit{Chlamydomonas reinhardtii} cells carrying the pHAWRA-Algal-1.0 plasmid at an initial density of $10^5$ cells/mL.
\end{enumerate}

\subsection{Quantum-Optical Synchronization in Liquid Culture}
\begin{enumerate}
    \item \textbf{Agitation:} Maintain constant stirring at 120 rpm to ensure uniform light exposure and $\text{CO}\_2$ diffusion.
    \item \textbf{Optical Density (OD) Monitoring:} Measure $OD\_{750}$ every 4 hours. The BioOS adjusts the LED intensity to maintain a constant "Effective Photon Flux" per cell.
    \item \textbf{Fidelity Measurement:} Use a flow-cytometry interface to measure the excitonic fluorescence response of individual cells as they pass through the "Quantum Gate" laser zone.
\end{enumerate}

\section{SOP-007: BioOS Calibration and Metabolic Sync}
Once the plant is regenerated and established in soil, the BioOS must be calibrated to the specific metabolic profile of the individual.

\subsection{Light Sensitivity Sweep and Reporter Validation}
\begin{enumerate}
    \item \textbf{Optical Setup:} Place the plant in the HAWRA growth chamber equipped with the modulated LED array.
    \item \textbf{Expression Mapping:} Apply a range of blue light (450nm) intensities (0-200 $\mu mol/m^2s$) and measure the expression of the \textit{Luciferase} reporter gene via a CCD camera.
    \item \textbf{Transfer Function Generation:} Calculate the "Light-to-Protein" transfer function, which the BioOS uses to translate ARBOL \texttt{apply} commands into specific LED power levels.
\end{enumerate}

\subsection{Coherence Baseline and Quantum Characterization}
\begin{enumerate}
    \item \textbf{P700 Absorption:} Use a dual-wavelength spectrophotometer to measure the $P\_{700}$ absorption change ($\Delta A$) in response to actinic light pulses.
    \item \textbf{Decay Analysis:} Establish the baseline $T\_2$ coherence time by fitting the absorption decay curve to the Lindblad model.
    \item \textbf{Silica Shield Verification:} Confirm that the measured $T\_2$ is $> 35$ ps, indicating successful silica accumulation.
\end{enumerate}

\subsection{Circadian Synchronization}
\begin{enumerate}
    \item \textbf{Entrainment:} Subject the plant to a 12h/12h light/dark cycle for 7 days.
    \item \textbf{Clock Alignment:} Align the BioOS internal scheduler with the peak of the plant's \textit{CCA1} gene expression (the master circadian regulator).
    \item \textbf{Metabolic Profiling:} Record the diurnal starch accumulation profile to define the "Computational Window" for the Bio-SGD optimizer.
\end{enumerate}

\section{SOP-003: BSIM Compilation and Deployment}
This protocol describes the process of translating high-level ARBOL logic into executable BSIM bytecode and deploying it to a live HAWRA node.

\subsection{Logic Compilation}
\begin{enumerate}
    \item \textbf{Source Validation:} Run the ARBOL source through the semantic analyzer to ensure all biological constraints (e.g., maximum pulse intensity, metabolic limits) are respected.
    \item \textbf{Bytecode Generation:} Use the \texttt{arbol-compiler} to generate the \texttt{.bsim.json} file.
    \item \textbf{Simulation Pre-flight:} Execute the generated BSIM in the HAWRA-Sim "Digital Twin" to verify that the expected fidelity is $> 0.90$.
\end{enumerate}

\subsection{Physical Deployment}
\begin{enumerate}
    \item \textbf{Node Handshake:} Establish a secure connection between the BioOS controller and the HAWRA growth chamber.
    \item \textbf{Resource Reservation:} The BioOS kernel checks the current metabolic state of the plant (Starch levels, $\text{CO}\_2$ flux). If resources are sufficient, the task is queued.
    \item \textbf{Execution:} The controller executes the BSIM instructions, modulating the light sources and nutrient pumps in real-time according to the compiled sequence.
\end{enumerate}

\subsection{Telemetry and Feedback}
\begin{enumerate}
    \item \textbf{Real-time Monitoring:} Capture chlorophyll fluorescence and leaf temperature throughout the execution.
    \item \textbf{Fidelity Estimation:} Use the PhytoQMML estimator to calculate the real-time gate fidelity based on the measured physiological response.
    \item \textbf{Post-Execution Audit:} Save the execution log and the resulting fidelity metrics to the local audit trail for subsequent synchronization with the global reproducibility database.
\end{enumerate}

\section{SOP-004: Maintenance and Pruning of the Quantum Substrate}
As a living system, the HAWRA processor requires regular physical maintenance to ensure optimal computational performance.

\subsection{Leaf Age and Replacement Strategy}
\begin{enumerate}
    \item \textbf{Fidelity Mapping:} Every 30 days, the BioOS performs a full-leaf "Fidelity Map."
    \item \textbf{Senescence Detection:} Identify leaves where the peak gate fidelity has dropped below 0.85 due to natural aging or metabolic fatigue.
    \item \textbf{Pruning:} Using a sterile surgical blade, remove the identified leaf at the petiole base.
    \item \textbf{Regeneration Trigger:} Apply a specialized "Growth Hormone Pulse" (via the chemical stimulus module) to the nearest axillary bud to trigger the development of a fresh, high-fidelity replacement leaf.
\end{enumerate}

\subsection{Silica Surface Cleaning}
\begin{enumerate}
    \item \textbf{Dust Interference:} Airborne particulates on the leaf surface can scatter control light, reducing gate precision.
    \item \textbf{Cleaning Protocol:} Every 14 days, mist the leaves with a 0.01\% Tween-20 solution in deionized water, followed by a gentle air-dry using filtered, $N\_2$ gas.
    \item \textbf{Calibration Reset:} Following cleaning, perform a full "Optical Sensitivity Sweep" (as per SOP-002) to recalibrate the light-to-DNA transfer function.
\end{enumerate}

\section{SOP-005: Emergency De-Transformation (Biosafety)}
In the event of a biosafety breach or a "Metabolic Meltdown" that threatens the plant's survival, the HAWRA system includes a protocol for rapid de-transformation.

\subsection{The "Scorched Earth" Genetic Reset}
\begin{enumerate}
    \item \textbf{Trigger:} Apply a high-intensity, 365nm (UV-A) pulse for 60 seconds across the entire plant.
    \item \textbf{Mechanism:} This pulse triggers a synthetic "excision circuit" within the pHAWRA plasmid. A UV-inducible Cre-recombinase enzyme is expressed, which targets the \textit{loxP} sites flanking the quantum core and the silica transporter genes.
    \item \textbf{Outcome:} The synthetic genes are physically removed from the plant's genome, leaving only the backbone pCAMBIA sequence. The plant reverts to its wild-type metabolic state within 48 hours.
\end{enumerate}

\subsection{Chemical Containment}
\begin{enumerate}
    \item \textbf{Root Drench:} In case of UV-trigger failure, apply a 500 mL drench of 50 $\mu$M "HAWRA-Stop" solution (a competitive inhibitor of the synthetic SIT1 transporter).
    \item \textbf{Metabolic Arrest:} This solution immediately halts all silica transport, causing the Silica Shield to degrade naturally within 24 hours and effectively "turning off" the quantum computer without killing the host.
\end{enumerate}

\section{Advanced Operational and Maintenance SOPs}

\subsection{SOP-006: Metabiotic Scheduler Configuration}
This procedure details how to tune the BioOS scheduler for different computational loads.
\begin{enumerate}
    \item Access the BioOS configuration terminal.
    \item Set the \texttt{ATP\_RESERVE\_THRESHOLD} to 35\% for high-intensity logic tasks.
    \item Configure the \texttt{ROS\_QUENCH\_INTERVAL} to 500ms during the peak diurnal phase.
    \item Verify the scheduler's "Metabolic Feasibility Score" calculation against the HAWRA-Sim reference.
\end{enumerate}

\subsection{SOP-007: Mycorrhizal Network Handshake}
This SOP describes the process of establishing a quantum-biological link between two HAWRA nodes.
\begin{enumerate}
    \item Verify that both nodes are colonized by the \textit{Rhizophagus irregularis} fungal network.
    \item Trigger a "Chemical Ping" pulse from Node A using the jasmonic acid stimulus module.
    \item Monitor the root-level electrical potential in Node B for the "MTP-SYN" signal.
    \item Confirm the handshake by observing the cross-node synchronization of the CRY2 optogenetic clocks.
\end{enumerate}

\subsection{SOP-008: Epigenetic Weight Synchronization}
This procedure is used to synchronize learned PhytoQMML weights across the HAWRA forest.
\begin{enumerate}
    \item Initiate the "Weight Broadcast" mode on the lead node (the "Mother Tree").
    \item Monitor the methylation rates at the \textit{pHAWRA} promoter sites in the target nodes.
    \item Use the "Read-Back" reporter genes to verify that the weights have been successfully persisted in the methylome.
    \item Perform a local "Fidelity Validation" run to confirm the transfer success.
\end{enumerate}

\subsection{SOP-010: High-Temperature Thermal Management}
When leaf temperatures exceed 35$^{\circ}$C, the following thermal management protocol must be activated.
\begin{enumerate}
    \item Override the default stomatal control to force 100\% aperture.
    \item Activate the root-level "Chilled Nutrient Feed" system (set to 18$^{\circ}$C).
    \item Reduce the quantum gate frequency by 50\% to lower the internal $Q\_{\text{quantum}}$ heat generation.
    \item \textbf{Emergency Shutdown:} If temperature continues to rise above 40$^{\circ}$C, trigger an immediate \texttt{INT\_TS} and enter the "State Flush" sequence (SOP-012).
\end{enumerate}

\subsection{SOP-012: Quantum State Flush to Anthocyanin Buffer}
In the event of an emergency shutdown, the quantum state must be flushed to a biological buffer to prevent data loss.
\begin{enumerate}
    \item Map the density matrix $\rho$ to the spatial distribution of anthocyanin pigments in the leaf epidermis.
    \item Apply a 532nm (green) laser pulse to trigger the "Pigment Fixation" reaction.
    \item Verify the successful flush via the multi-spectral imaging sensor.
    \item Log the state-flush completion to the BioOS audit trail.
\end{enumerate}

\subsection{SOP-015: Silica Shield Surface Repair}
If the silica shield shows signs of degradation or non-uniformity (SOP-002), follow this repair protocol.
\begin{enumerate}
    \item Upregulate the \textit{SIT1} gene expression by applying a 10s "Silica Stimulus" pulse (405nm).
    \item Supplement the nutrient feed with 500 $\mu$M Orthosilicic acid ($H\_4SiO\_4$).
    \item Monitor the silica accumulation rate in real-time via the XRF scanner.
    \item Once the 5.0nm target thickness is restored, return to normal operation.
\end{enumerate}

\subsection{SOP-020: Decommissioning and Genetic De-activation}
This SOP ensures the safe decommissioning of a HAWRA node at the end of its life cycle.
\begin{enumerate}
    \item Perform a final "Epigenetic Reset" (SOP-005) to erase all PhytoQMML weights.
    \item Trigger the UV-inducible Cre-recombinase "Scorched Earth" excision circuit.
    \item Harvest the silica-rich leaf tissue for material recycling.
    \item Compost the remaining organic matter in a specialized HAWRA-safe containment facility.
\end{enumerate}

\subsection{SOP-021: Quantum Gate Calibration (Z-Gate)}
This procedure calibrates the phase shift ($\phi$) of the Z-gate by modulating the localized refractive index of the thylakoid membrane.
\begin{enumerate}
    \item Apply a reference Hadamard gate $H$ to initialize the exciton in a superposition state $|+\rangle$.
    \item Modulate the SIT1 transporter activity to create a localized silica gradient across the P700 complex.
    \item Measure the resulting phase shift using Ramsey interferometry via the fluorescence reporter.
    \item Adjust the SIT1 drive current until the measured phase shift matches the target $\phi = \pi/2$ (for a $Z^{1/2}$ gate).
\end{enumerate}

\subsection{SOP-022: Mycorrhizal Data Routing and Packet Recovery}
This SOP describes how the BioOS handles packet loss in the fungal network.
\begin{enumerate}
    \item Monitor the "Pulse-to-Noise" ratio (PNR) of the incoming calcium wave signals.
    \item If PNR drops below 10dB, the BioOS initiates a "Request for Retransmission" (MTP-REQ) to the source node.
    \item If the fungal hyphae is physically damaged, the BioOS automatically searches for an alternative "Biotic Route" through the nearest healthy neighboring node.
    \item Log the network topology change in the global HAWRA-Mesh status registry.
\end{enumerate}

\subsection{SOP-023: Bio-SGD Model Retraining and Hyperparameter Tuning}
This procedure is used to retrain a PhytoQMML model when the environment shifts significantly (e.g., seasonal change).
\begin{enumerate}
    \item Capture a 24-hour "Metabolic Trace" of the plant's response to the new environmental conditions.
    \item Update the Bio-SGD cost function coefficients ($\alpha, \beta, \gamma$) to prioritize metabolic health over speed if $\text{CO}\_2$ levels are low.
    \item Execute a 100-epoch "Warm-up Run" using the historical weights stored in the L2 (Methylation) storage.
    \item Validate the new model accuracy against the classical benchmark data in the HAWRA-Sim cloud.
\end{enumerate}

\subsection{SOP-024: Genetic Firewall Audit and Malware Scanning}
This procedure is used to verify the integrity of the host genome and the pHAWRA plasmid.
\begin{enumerate}
    \item \textbf{Sequence Verification:} Perform a targeted nanopore sequencing run on the pHAWRA plasmid region.
    \item \textbf{Signature Matching:} Compare the sequence against the "Golden Master" hash stored in the BioOS kernel.
    \item \textbf{Decoy Check:} Verify the activity of the Transcription Factor Decoys (Layer 1) by applying a synthetic "Probe Peptide."
    \item \textbf{Reporting:} If any unauthorized mutations or insertions are detected, trigger an immediate \texttt{INT\_GS} and follow the sanitization protocol in SOP-020.
\end{enumerate}

\section{Conclusion}
The rigorous application of these SOPs ensures that HAWRA remains a safe, reproducible, and scalable platform for carbon-negative quantum computing.
