\chapter{Glossary of Terms}

\begin{description}
    \item[AIR (ARBOL Intermediate Representation):] The low-level representation of ARBOL code used by the compiler for optimization and metabolic validation.
    \item[ARBOL:] The domain-specific language for programming living quantum computers.
    \item[ATP (Adenosine Triphosphate):] The primary energy currency of the cell, used to power both biological processes and quantum gate operations.
    \item[BIVT (Biological Interrupt Vector Table):] A table used by the BioOS to map biological stress events to high-priority interrupt handlers.
    \item[Bio-SGD:] Biological Stochastic Gradient Descent; an optimization algorithm that incorporates metabolic and quantum constraints.
    \item[BioOS:] Biological Operating System; the software layer that manages the host plant's physiology and quantum operations.
    \item[Biotic QEC:] Quantum Error Correction techniques that leverage the natural redundancy and self-healing properties of biological systems.
    \item[BSIM (Bio-Simulation Instruction Map):] The executable bytecode format for HAWRA nodes, mapping logical instructions to biological stimuli.
    \item[CCT (Carbon-Compute Token):] A digital asset representing the combined value of carbon sequestered and computation performed by a HAWRA node.
    \item[Calvin Cycle:] The set of chemical reactions that take place in chloroplasts during photosynthesis to fix carbon into starch.
    \item[Circadian Clock:] The internal 24-hour biological clock that regulates the plant's metabolism and the BioOS scheduling profile.
    \item[CpG Site:] Regions of DNA where a cytosine nucleotide is followed by a guanine nucleotide; the primary site for DNA methylation in plants.
    \item[CRY2-PIF3:] A blue-light sensitive optogenetic system used as the primary "interrupt" and "write" interface in HAWRA.
    \item[Decoherence:] The loss of quantum coherence in a system due to interaction with the environment (e.g., thermal noise).
    \item[EDF-MA:] Earliest Deadline First with Metabolic Awareness; the real-time scheduling algorithm used by the BioOS.
    \item[Epigenetic Weight Persistence:] The long-term storage of model weights in the methylation patterns of the plant's genome.
    \item[Exciton:] A mobile concentration of energy in a crystal or protein, formed by an excited electron and the "hole" it left behind.
    \item[FMO Complex:] Fenna-Matthews-Olson complex; a protein that demonstrates highly efficient excitonic energy transfer.
    \item[Gaia-OS:] The theoretical planetary-scale operating system that manages the Earth's biosphere through the Global HAWRA Mesh.
    \item[Genetic Firewall:] A multi-layered security system designed to prevent unintended genetic modifications or viral attacks on the HAWRA node.
    \item[HAWRA:] Hardware-Agnostic Wetware-Reliant Architecture; the core paradigm for living quantum computing.
    \item[HAWRA-Sim:] A multiphysics simulation engine that couples quantum dynamics with metabolic and environmental models.
    \item[Histone Modification:] Chemical changes to the proteins around which DNA is wrapped; used as a "cache" memory in PhytoQMML.
    \item[L-thread (Living Thread):] A sequence of instructions executed within a specific metabolic budget in the BioOS.
    \item[Learning Efficiency ($\eta\_L$):] A thermodynamic metric measuring the metabolic energy cost per bit of information learned.
    \item[Lindblad Master Equation:] The mathematical framework used to describe the evolution of a quantum system in a noisy environment.
    \item[M3 (Metabolic Memory Management):] The BioOS component responsible for managing epigenetic and metabolic storage resources.
    \item[Metabiotic Computing:] A paradigm where the computational substrate is a living organism, and the logic is integrated with its life processes.
    \item[Methylome:] The complete set of DNA methylation patterns in a cell or organism.
    \item[MTP (Mycorrhizal Transmission Protocol):] The communication protocol used for data exchange over fungal networks.
    \item[Mycorrhizal Mesh:] The distributed network of plants and fungi that enables large-scale communication and federated learning.
    \item[P700:] The primary electron donor in Photosystem I, serving as the physical site for quantum logic operations.
    \item[pHAWRA:] The synthetic plasmid that encodes the BioOS and the necessary biological modifications for quantum computing.
    \item[Photoinhibition:] The reduction in a plant's capacity for photosynthesis due to exposure to excessive light.
    \item[PhytoQMML:] Phyto-synthetic Quantum Machine Learning; the framework for AI on living substrates.
    \item[PQPE:] Phyto-synthetic Quantum Processing Entity; a single leaf or plant configured as a computing node.
    \item[Quantum Kernel:] The mapping of classical data into a high-dimensional Hilbert space using the natural excitonic dynamics of the chloroplast.
    \item[ROS (Reactive Oxygen Species):] Chemically reactive molecules containing oxygen; a byproduct of photosynthesis that can cause metabolic stress.
    \item[Silica Shield:] A biomineralized dielectric layer that reduces decoherence by screening the PQPE from environmental noise.
    \item[SIT1:] Silicon Transporter 1; the gene targeted for over-expression to build the silica shield.
    \item[Thylakoid:] The membrane system inside chloroplasts where the quantum-active P700 complexes are located.
    \item[Vibronic Coupling:] The interaction between electronic transitions and vibrational modes of the protein scaffold, often used to enhance coherence.
    \item[Anthocyanin Buffer:] A biological storage medium that uses pigment distribution to store quantum states during emergency shutdowns.
    \item[BFL (Biological Federated Learning):] The process of training machine learning models across a network of living nodes without sharing raw data.
    \item[Biotic-Huffman:] A compression algorithm optimized for the biological transmission of weight updates via calcium waves.
    \item[Chloroplast:] The organelle in plant and algal cells where photosynthesis and quantum logic operations take place.
    \item[Coherence Window:] The period during which a quantum state remains stable enough for logical operations.
    \item[Digital Twin:] A high-fidelity numerical simulation of a specific HAWRA node, used for pre-flight validation of code.
    \item[GDZ (Global Deployment Zone):] A specific ecological region targeted for the rollout of HAWRA technology.
    \item[GQHO (Genetic-Quantum Hybrid Optimizer):] An optimization algorithm used to find the best pulse sequences for Quantum Kernel mapping.
    \item[LHS (Latin Hypercube Sampling):] A statistical method for generating a near-random sample of parameter values from a multidimensional distribution.
    \item[MGC (Metabolic Garbage Collector):] The BioOS component that reclaims epigenetic and metabolic resources that are no longer needed.
    \item[Metabolic Kill-Switch:] A safety mechanism that triggers programmed senescence if a HAWRA node is removed from its authorized environment.
    \item[NESS (Nonequilibrium Steady State):] A state in which a system is continuously dissipating energy to maintain its internal order.
    \item[Optogenetics:] A biological technique that uses light to control the activity of specific genes or proteins.
    \item[PBFT (Practical Byzantine Fault Tolerance):] A consensus algorithm modified for use in the mycorrhizal mesh, where voting power is tied to metabolic health.
    \item[RBN (Recurrent Biotic Network):] A neural network architecture that uses metabolic feedback loops to process temporal data.
    \item[Sobol Indices:] A set of sensitivity measures used to determine the relative importance of different input parameters in the HAWRA system.
    \item[T-CNN (Thylakoid CNN):] A convolutional neural network architecture that leverages the spatial layout of P700 complexes for image and pattern processing.
\end{description}
