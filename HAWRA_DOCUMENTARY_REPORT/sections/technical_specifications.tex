\chapter{Technical Specifications}

\section{PQPE Node Specifications (Version 2.0)}
This section provides the definitive technical specifications for a single HAWRA PQPE node based on the \textit{Ficus elastica} substrate.

\subsection{Physical Characteristics}
\begin{itemize}
    \item \textbf{Substrate Species:} \textit{Ficus elastica} (modified via pHAWRA).
    \item \textbf{Typical Leaf Surface Area:} $150 - 250$ $cm^2$.
    \item \textbf{Stomatal Density:} $25,000$ $cm^{-2}$ (Abaxial).
    \item \textbf{Chlorophyll Content:} $45 - 60$ $\mu g/cm^2$.
    \item \textbf{Silica Shield Thickness:} $5.0 \pm 0.5$ nm (Uniformity $> 98\%$).
\end{itemize}

\subsection{Quantum Performance Metrics}
\begin{itemize}
    \item \textbf{Number of Logical Qubits:} 64 per leaf (scalable via mycorrhizal mesh).
    \item \textbf{Peak Gate Fidelity ($F$):} $0.952$ (Hadamard), $0.941$ (CNOT).
    \item \textbf{Coherence Time ($T\_2$):} $150 \mu$s at $25^\circ$C (Metabiotic Stabilization).
    \item \textbf{Gate Speed:} $10 - 50$ ns (Optical Pulse Width).
    \item \textbf{Measurement Fidelity:} $0.985$ (Fluorescence-based readout).
\end{itemize}

\section{BioOS Controller Specifications}
The BioOS controller is the hardware interface that manages the PQPE node.

\subsection{Optical Interface Unit (OIU)}
\begin{itemize}
    \item \textbf{Excitation Source:} Femtosecond Pulsed Laser (Ti:Sapphire).
    \item \textbf{Wavelength Range:} $680 - 720$ nm (Qubit Manipulation), $450$ nm (Genetic Trigger).
    \item \textbf{Spatial Resolution:} $5 \mu m$ (Digital Micromirror Device).
    \item \textbf{Peak Power:} $500$ mW.
\end{itemize}

\subsection{Physiological Sensing Unit (PSU)}
\begin{itemize}
    \item \textbf{PAM Fluorometer:} Multi-frequency pulse modulation for $F\_v/F\_m$ monitoring.
    \item \textbf{IR Gas Analyzer (IRGA):} Differential $CO\_2/H\_2O$ measurement (Sensitivity $0.1$ ppm).
    \item \textbf{Leaf Temperature Sensor:} Micro-bolometer array (Resolution $0.05^\circ$C).
\end{itemize}

\section{ARBOL Language Capabilities}
\begin{itemize}
    \item \textbf{Compiler:} BSIM v1.2 (Targeting BioOS Kernel 4.0).
    \item \textbf{Optimizations:} Circadian Scheduling, ROS-Aware Pulse Shaping, Genetic Pipelining.
    \item \textbf{Supported Gates:} H, X, Y, Z, CNOT, SWAP, CZ, Toffoli (Experimental).
    \item \textbf{Formal Verification:} Static ATP budgeting, Deadlock prevention.
\end{itemize}

\section{Metabiotic Scaling Laws}
The following power-law relationships define the scaling of the HAWRA system as the biological host grows.

\begin{table}[H]
    \centering
    \begin{tabular}{lcl}
        \toprule
        \textbf{Relationship} & \textbf{Exponent ($\alpha$)} & \textbf{Physical Driver} \\
        \midrule
        Qubits vs. Leaf Area & $0.76$ & Metabolic supply limit \\
        $T\_2$ vs. Silica Uniformity & $1.42$ & Dielectric screening efficiency \\
        Error Rate vs. ATP Level & $-2.15$ & Lindblad dissipation coupling \\
        Mesh Bandwidth vs. Hyphal Density & $0.88$ & Calcium wave propagation \\
        \bottomrule
    \end{tabular}
    \caption{Empirical Scaling Laws for HAWRA-based Systems.}
    \label{tab:scaling_laws}
\end{table}

\section{Mycorrhizal Network Parameters (MTP v1.1)}
The fungal network provides the communication backbone for the Global HAWRA Mesh.

\subsection{Transmission Characteristics}
\begin{itemize}
    \item \textbf{Carrier Signal:} Calcium wave modulation ($Ca^{2+}$ pulses).
    \item \textbf{Propagation Velocity:} $5 - 15$ mm/min.
    \item \textbf{Bandwidth:} $1.2$ kbps (per hyphal link).
    \item \textbf{Latency:} $120 - 300$ seconds (intra-node), $1 - 4$ hours (inter-tree).
    \item \textbf{Error Rate (BER):} $10^{-4}$ (Before FEC).
\end{itemize}

\subsection{Network Topology}
The HAWRA mesh employs a "Living Small-World" topology, where most nodes are connected to their neighbors, with a few "hub" trees providing long-distance links through deep fungal pathways.

\section{Nutrient Supplement: HAWRA-A Data Sheet}
The HAWRA-A solution is critical for maintaining the Silica Shield and the enhanced P700 concentration.

\begin{itemize}
    \item \textbf{Silica Concentration:} $25$ mM (as Monosilicic acid).
    \item \textbf{Quantum Buffers:} $500 \mu$M Anthocyanin precursors.
    \item \textbf{Metabolic Boosters:} $10$ mM ATP-synthase co-factors.
    \item \textbf{pH:} $5.8 \pm 0.1$.
    \item \textbf{Electrical Conductivity (EC):} $1.8$ mS/cm.
\end{itemize}

\section{Environmental Requirements}
To maintain peak computational performance, the HAWRA node must be kept within the following parameters:
\begin{itemize}
    \item \textbf{Ambient Temperature:} $22 - 28^\circ$C.
    \item \textbf{Relative Humidity:} $60 - 75\%$.
    \item \textbf{Light Intensity (Background):} $300 - 500$ $\mu mol/m^2s$ (PAR).
    \item \textbf{Nutrient Supply:} HAWRA-A (Proprietary metabolic supplement).
\end{itemize}
