\documentclass[a4paper,12pt]{report}
\usepackage[utf8]{inputenc}
\usepackage[T1]{fontenc}
\usepackage{geometry}
\usepackage{graphicx}
\usepackage{hyperref}
\usepackage{listings}
\usepackage{xcolor}
\usepackage{enumitem}
\usepackage{float}
\usepackage{caption}
\usepackage{subcaption}
\usepackage{amsmath}
\usepackage{amssymb}

% Page Geometry
\geometry{margin=2.5cm}

% Hyperlink Setup
\hypersetup{
    colorlinks=true,
    linkcolor=blue,
    filecolor=magenta,      
    urlcolor=cyan,
    pdftitle={HAWRA Comprehensive Documentation},
    pdfauthor={HAWRA Team},
}

% Code Listing Style
\definecolor{codegreen}{rgb}{0,0.6,0}
\definecolor{codegray}{rgb}{0.5,0.5,0.5}
\definecolor{codepurple}{rgb}{0.58,0,0.82}
\definecolor{backcolour}{rgb}{0.95,0.95,0.92}

\lstdefinestyle{mystyle}{
    backgroundcolor=\color{backcolour},   
    commentstyle=\color{codegreen},
    keywordstyle=\color{magenta},
    numberstyle=\tiny\color{codegray},
    stringstyle=\color{codepurple},
    basicstyle=\ttfamily\footnotesize,
    breakatwhitespace=false,         
    breaklines=true,                 
    captionpos=b,                   
    keepspaces=true,                 
    numbers=left,                   
    numbersep=5pt,                  
    showspaces=false,                
    showstringspaces=false,
    showtabs=false,                  
    tabsize=2
}

\lstset{style=mystyle}

% Title Page
\title{
    \Huge \textbf{HAWRA}\\
    \huge Programmable Quantum-Physiological Plant Entity\\
    \vspace{1cm}
    \Large Comprehensive Academic Documentation\\
    \vspace{1cm}
    \large Version 1.0\\
    \vspace{2cm}
    \textbf{The HAWRA Team}\\
    \textit{Move37 / HAWRA Initiative}\\
    \date{\today}
}
\author{}

\begin{document}

\maketitle

\tableofcontents
\listoffigures
\newpage

\chapter{Introduction}

\section{Project Overview}
HAWRA (Hardware-Agnostic Wetware-Reliant Architecture) is an interdisciplinary initiative aimed at realizing in vivo biological computation within \emph{Ficus elastica}. The project unifies quantum biology, synthetic biology, and computer science to create a "wetware" computer capable of processing information through bio-quantum states (excitons in chloroplasts) regulated by genetic circuits.

The core of this system is the \textbf{PQPE} (Phyto-synthetic Quantum Processing Entity), a genetically modified plant organism that acts as the hardware substrate. This entity is orchestrated by the \textbf{BioOS} (Biological Operating System) and programmed via the \textbf{Arbol} language.

\section{Objectives}
The primary objectives of HAWRA are:
\begin{itemize}
    \item \textbf{PhytoQMML}: To implement "Phyto Quantum Mathematical-Metabolic Learning," a paradigm where optimization and learning are embodied by biological processes.
    \item \textbf{PQPE Construction}: To engineer a Programmable Quantum-Physiological Plant Entity using a standardized genetic backbone (pHAWRA plasmid).
    \item \textbf{Unified Simulation}: To provide a complete software stack (Arbol language, BioOS, Simulator) that models the interaction between biological, quantum, and environmental states with high fidelity.
\end{itemize}

\section{Scope}
This document covers the entire technical stack, from the high-level Domain Specific Language (Arbol) to the low-level genetic implementation (pHAWRA plasmid) and the validation results obtained through the unified simulator.

\chapter{System Architecture}

\section{Global Architecture}
The HAWRA system is structured around several key components that facilitate the flow from code to biological execution (simulated or real). The architecture ensures a strict control hierarchy where Arbol is the sole entry point.

\begin{itemize}
    \item \textbf{User Level}: Interaction via Arbol IDE or CLI.
    \item \textbf{Compiler Level}: Translation of Arbol code into Bio-Simulation (BSIM) JSON format.
    \item \textbf{OS Level (BioOS)}: Interpretation of BSIM instructions into physical stimuli (Light, EM fields).
    \item \textbf{Hardware Level (PQPE)}: Biological execution in \emph{Ficus elastica} chloroplasts.
\end{itemize}

\begin{figure}[h]
    \centering
    % Placeholder for architecture diagram if available
    \includegraphics[width=0.9\textwidth, keepaspectratio]{../results/hawra_architecture.png}
    \caption{Global Architecture of the HAWRA System}
\end{figure}

\section{Components}
\subsection{PQPE (Phyto-synthetic Quantum Processing Entity)}
The PQPE is the biological hardware. It utilizes the P700 reaction center of Photosystem I as a qubit, encapsulated in a silica nano-cage (synthesized via Lsi1/SIT1) to preserve quantum coherence.

\subsection{BioOS (Biological Operating System)}
BioOS acts as the bridge between the digital and biological worlds. It manages:
\begin{itemize}
    \item Translation of quantum gates into light pulses and electromagnetic fields.
    \item Regulation of the plant's metabolic state (homeostasis).
    \item Readout of computation results via bioluminescent reporters (Luciferase).
\end{itemize}

\chapter{The Arbol Language}

\section{Overview}
Arbol (A-life Regulation and Bio-computation Orchestration Language) is a Domain-Specific Language (DSL) designed to express bio-quantum algorithms. It abstracts complex biological regulatory networks and quantum operations into high-level primitives.

\section{Syntax and Grammar}
Arbol supports standard quantum operations (H, X, Y, Z, CNOT) and biological constructs (Logical Qubits linked to promoters).

\begin{lstlisting}[language=Python, caption=Example Arbol Script (PhytoQMML)]
circuit phytoqmmml_demo {
    // Define qubits linked to biological promoters
    LOGICAL_QUBIT q1 [1] IS { activator: p1 };
    LOGICAL_QUBIT q2 [1] IS { repressor: p2 };

    // Apply quantum operations
    H(q1);
    CNOT(q1, q2);

    // Measure state
    measure q2;
}

run phytoqmmml_demo;
\end{lstlisting}

\section{Compiler and BSIM}
The Arbol compiler transforms the source code into a JSON-based format called BSIM (Bio-Simulation). This format contains:
\begin{itemize}
    \item \texttt{metadata}: Version and source info.
    \item \texttt{instructions}: An ordered list of commands (\texttt{INITIALIZE}, \texttt{QUANTUM\_OP}, \texttt{MEASURE}).
\end{itemize}

\chapter{Biological Implementation (pHAWRA)}

\section{The pHAWRA Plasmid}
The genetic backbone of the PQPE is the pHAWRA plasmid. It contains a modular arrangement of gene cassettes designed for specific functions.

\begin{table}[H]
\centering
\begin{tabular}{|l|l|l|}
\hline
\textbf{Gene} & \textbf{Function} & \textbf{Role in BioOS} \\ \hline
\texttt{psaA} & Photosystem I Core & Qubit Register (P700) \\ \hline
\texttt{CRY2} & Cryptochrome & Quantum Gate Interface (Blue Light) \\ \hline
\texttt{Lsi1/SIT1} & Silicon Transporter & Quantum Isolation (Silica Cage) \\ \hline
\texttt{Luc} & Luciferase & Readout (Bioluminescence) \\ \hline
\texttt{HSP70} & Heat Shock Protein & Thermal Error Correction \\ \hline
\texttt{PEPC} & PEP Carboxylase & Energy Management (CAM) \\ \hline
\end{tabular}
\caption{pHAWRA Genetic Components}
\end{table}

\section{Quantum Isolation Strategy}
To maintain quantum coherence in a wet, warm environment, HAWRA employs a silica nano-cage strategy. The expression of \texttt{Lsi1} and \texttt{SIT1} facilitates the deposition of a silica shell around the thylakoid membranes, shielding the P700 qubits from thermal noise and decoherence.

\chapter{Unified Simulator and Physics}

\section{Simulation Model}
The Unified Simulator models the system using the Lindblad Master Equation for open quantum systems:

\begin{equation}
\frac{d\rho}{dt} = -\frac{i}{\hbar} [H, \rho] + \sum_k \gamma_k \left( L_k \rho L_k^\dagger - \frac{1}{2} \{L_k^\dagger L_k, \rho\} \right)
\end{equation}

Where:
\begin{itemize}
    \item $\rho$ is the density matrix of the qubit state.
    \item $H$ is the Hamiltonian describing the system's energy and control fields.
    \item $L_k$ are the Lindblad operators representing decoherence channels.
\end{itemize}

\section{Spectral Sweep}
The simulator accounts for the wavelength-dependent response of the biological components. A spectral sweep analysis determines the optimal wavelengths for controlling specific genes (e.g., 450nm for CRY2, 650nm for PhyB).

\begin{figure}[h]
    \centering
    \includegraphics[width=0.8\textwidth, keepaspectratio]{../03_unified_simulator/results/spectral_sweep.png}
    \caption{Spectral Sweep Analysis}
\end{figure}

\chapter{Results and Validation}

\section{Quantum Coherence}
Simulations demonstrate that the engineered environment (silica confinement + protein stabilization) allows for extended quantum coherence times.

\begin{figure}[h]
    \centering
    \includegraphics[width=0.8\textwidth, keepaspectratio]{../03_quantum_simulation/results/qubit_coherence_validation.png}
    \caption{P700 Coherence Decay Simulation}
\end{figure}

\section{Pipeline Verification}
The entire pipeline, from Arbol code to simulated readout, has been validated. The "One-Shot Pipeline" ensures:
\begin{enumerate}
    \item Successful compilation of Arbol scripts.
    \item Correct generation of BSIM instructions.
    \item Accurate simulation of quantum dynamics.
    \item Generation of validation plots and data bundles.
\end{enumerate}

\chapter{Conclusion}

The HAWRA project establishes a comprehensive framework for vegetal quantum computing. By defining a rigorous language (Arbol), a valid genetic implementation (pHAWRA), and a unified simulator, we provide the necessary tools to explore the frontier of "wetware" computing. The results confirm the theoretical viability of the approach, showing coherence times >1ps and robust system programmability, paving the way for wet-lab experimentation in 2026.

\bibliographystyle{plain}
\begin{thebibliography}{9}
\bibitem{hawra_formal}
HAWRA Team. \textit{HAWRA-PQPE Formal Model}. 2025.
\bibitem{phytoqmml}
HAWRA Team. \textit{Phyto Quantum Mathematical-Metabolic Learning Formalization}. 2025.
\bibitem{arbol_spec}
HAWRA Team. \textit{Arbol Language Specification}. 2025.
\end{thebibliography}

\end{document}
